% Thesis defense slides
% June 29 2017
% Dr Dale Bowman Dr Saunak Sen Dr Su Chen
% Topic Modeling

%%%%%%%%%%%%%%%%%%%% SETUP %%%%%%%%%%%%%%%%%%%%%%%%%
\documentclass{beamer}

\newcommand{\bw}{\mbox{\boldmath $w$}}
\newcommand{\bz}{\mbox{\boldmath $Z$}}
\newcommand{\bt}{\mbox{\boldmath $\theta$}}
\newcommand{\bp}{\mbox{\boldmath $\phi$}}

\mode<presentation>
{
\usetheme{Warsaw}
\setbeamercovered{transparent}
}

\usepackage{tikz}
\usetikzlibrary{shapes,arrows}
\usetikzlibrary{positioning}
\usetikzlibrary{shapes.geometric,arrows,fit,calc,positioning,automata,}
\usepackage[english]{babel}
\usepackage[utf8]{inputenc}
\usepackage{times}
\usepackage{amsmath}
\usepackage[T1]{fontenc}
\graphicspath{ {images/} }
\setbeamerfont{footnote}{size=\tiny}
%%%%%%%%%%%%%%%%%%%% TITLE PAGE  %%%%%%%%%%%%%%%%%%%%%%%%%
\title[Topic Modeling] % (optional, use only with long paper titles)
{Topic Modeling}
\subtitle
{Comparison of methods for choosing an optimum value for the number of topics in an LDA model} 
\author{Patricia Jean Goedecke} % (optional, use only with lots of authors)
\institute{Supervised by Dr. Dale Bowman \\Mathematical Sciences, Statistics\\University of Memphis} % (optional, but mostly needed)
\subject{Topic Modeling}
% If you wish to uncover everything in a step-wise fashion, uncomment
% the following command: 
%\beamerdefaultoverlayspecification{<+->}

\begin{document}
\begin{frame}
\titlepage
\end{frame}

%%%%%%%%%%%%%%%%%%%% OUTLINE %%%%%%%%%%%%%%%%%%%%%%%%%
\begin{frame}{Outline}
\tableofcontents %[pausesections]
% You might wish to add the option[pausesections]
\end{frame}


%%%%%%%%%%%%%%%%%%%% INTRODUCTION %%%%%%%%%%%%%%%%%%%%%%%%%
\section{Introduction}
\begin{frame}{Introduction}
\end{frame}



\begin{frame}{Introduction: What is topic modeling?}
\begin{itemize}
\item
Unsupervised dimension reduction or clustering technique\\
\vspace{.3cm}
\item Assumes latent topics joining documents to words orthogonally\\
\vspace{.3cm}
\item Originally used with large corpuses of text data\\
\vspace{.3cm}
\item Recently applied to other big data including biomedical data sets 
\end{itemize}
\end{frame}


\begin{frame}{Introduction: What is topic modeling?}
\begin{figure}[htb!]
\includegraphics[width=8cm,height=5cm]{LDA.png}
\end{figure}
\footnotetext[0]{Image credit Blei 2011}
\end{frame}


%%%%%%%%%%%%%%%%%%%% PRE-LDA %%%%%%%%%%%%%%%%%%%%%%%%%
\section{Methods leading to Latent Dirichlet Allocation}
\begin{frame}{Methods leading to Latent Dirichlet Allocation}
\end{frame}


% Deerwester et al. (1990) 
%%subsection{Latent Semantic Analysis and singular value decomposition}
\begin{frame}{Latent Semantic Analysis (LSA)}
\begin{columns}
\column{.25\textwidth}
\begin{figure}[htb!]
\includegraphics[width=2.75cm,height=2.4cm]{Deerwester.jpg}
\end{figure}

\column{.7\textwidth}
\begin{itemize}
\item Deerwester et al. (1990) \\
\item unsupervised text modeling technique
\item term by document matrix 
\item orthogonal factors 
\item New indexing method found "promising" 
\end{itemize}
\end{columns}
\footnotetext[1]{Photographic Archive, [apf12345], Special Collections Research Center, \\University of Chicago Library.}
\end{frame}

% pLSI Hofmann(1999) 
%subsection{Probabilistic Latent Semantic Analysis}
\begin{frame}{Probabilistic latent semantic indexing (pLSI, pLSA)}
\begin{columns}
\column{.3\textwidth}
\begin{figure}[htb!]
\includegraphics[width=2.75cm,height=2.75cm]{pLSI.jpg}
\end{figure}
\column{.7\textwidth}
\begin{itemize}
\item Hofmann (1999)
\item uses a mixture decomposition
\item maximum likelihood model
\item tempered expectation–maximization (EM) iterative algorithm 
\end{itemize}
\end{columns}
\footnotetext[2]{Image credit: http://www.cnblogs.com/yuzhung/archive.html }
\end{frame}


% NNMF Lee Seung (1999); Ding (2008)
\begin{frame}{Non-negative matrix factorizaton (NMF)}
\begin{columns}
\column{.4\textwidth}
\begin{figure}[htb!]
\includegraphics[width=4cm,height=3cm]{NMF.jpg}
\end{figure}
\column{.6\textwidth}
\begin{itemize}
\item Lee \&\ Seung (1999)
\item parts-based representation
\item described as a neural network
\item found by Ding (2008) to optimize the same objective function as pLSI
\end{itemize}
\end{columns}
\footnotetext[1]{Image credit Nature.com Lee \&\ Seung (1999)}
\end{frame}

%%%%%%%%%%%%%%%%%%%% LDA %%%%%%%%%%%%%%%%%%%%%%%%%
\section{LDA and current trends}
\begin{frame}{LDA and current trends}
\end{frame}

% Blei, Ng and Jordan (2003) 
\begin{frame}{Latent Dirichlet allocation (LDA)}
\begin{columns}
\column{.3\textwidth}
\begin{figure}[htb!]
\includegraphics[width=3cm,height=4cm]{Blei.jpg}
\end{figure}
\column{.7\textwidth}
\begin{itemize}
\item Blei, Ng and Jordan (2003).
\item document is finite mixture over topics
\item topic is infinite mixture over probability distribution
\item modeled using variational methods, EM algorithm
\end{itemize}
\end{columns}
\footnotetext[2]{Photo credit: Denise Applewhite}
\end{frame}

%%%%%%%%%%%%%%%%%%%% Current Trends %%%%%%%%%%%%%%%%%%%%%%%%%



\begin{frame}{Expanding applications}
\begin{figure}[htb!]
\includegraphics[width=8cm,height=5cm]{EMR-cartoon.jpg}
\end{figure}
\footnotetext[1]{ cldeanrobles.files.wordpress.com }
\end{frame}

\begin{frame}{Topic coherence \&\ selecting the number of topics}
\begin{columns}
\column{.3\textwidth}
\begin{figure}[htb!]
\includegraphics[width=3cm,height=3cm]{polyaurn.png}
\end{figure}
\column{.7\textwidth}
\begin{itemize}
\item Pointwise Mutual Information (PMI), Newman et al. (2010)
\item Generalized P\'{o}lya Urn model, Mimno et al. (2011) 
\end{itemize}
\end{columns}
\footnotetext[1]{http://fac.ksu.edu.sa/raguech/blog/173205}
\end{frame}

%%%%%%%%%%%%%%%%%%%% Topic Modeling with LDA %%%%%%%%%%%%%%%%%%%%%%%%%
\section{Topic modeling using latent Dirichlet allocation}
\begin{frame}{Topic modeling using latent Dirichlet allocation}
\end{frame}

\begin{frame}{de Finetti's theorem and exchangeability}
\begin{columns}
\column{.3\textwidth}
\begin{figure}[htb!]
\includegraphics[width=3cm,height=4cm]{BagofWords.jpg}
\end{figure}
\column{.7\textwidth}
\begin{itemize}
\item Exchangeable random variables are conditionally independent,
conditioned on some latent variable
\item In topic modeling, words are conditionally independent,
conditioned on topic
\end{itemize}
\end{columns}
\footnotetext[1]{http://www.novuslight.com.html}
\end{frame}

%%%%%%%%%%%%%%%%%%%% Document generation %%%%%%%%%%%%%%%%%%%%%%%%%
\begin{frame}{LDA hypothetical document generation}
Representation $\sim$
\begin{enumerate}
\item  A corpus consists of $D$ documents\vspace{.5cm}
\item  A document consists of $N$ words from vocabulary of size $V$
\vspace{.5cm}
\item Each word $w_{n}$ is chosen from one of $K$ topics.
\end{enumerate}
\end{frame}

\begin{frame}{LDA hypothetical document generation}
\begin{enumerate}
\item Number of words (N) for document selected from Poisson, parameter {\mbox{\boldmath $\xi$}}. \\
\item Multinomial probability vector {\mbox{\boldmath $\theta$}}  length $K$ selected from Dirichlet, parameter {\mbox{\boldmath $\alpha$}}
\item The$n$th word is chosen from the following process:
\end{enumerate}
\begin{itemize}
\item Choose a topic, $Z_n$, from $Mult(1,\theta)$
\item Choose a $V \times 1$ vector $\phi$ from $Dir(\beta_{Z_n})$
\item Choose a word, $w_n$ from $Mult(1, \phi)$
\end{itemize}
\end{frame}

\begin{frame}{LDA hypothetical document generation}
In this model
\begin{itemize}
\item $w_{nj}$=1 if the $n$th word is the $j$th in the vocabulary
\item $Z_{ni}$ = 1 if the $n$th word is in topic $i$
\item $w_{nj}$ is observed
\item $Z_{ni}$ is unobserved
\item $\phi_{ij}$ =$ p(w_{nj}$ = $1|z_{ni}=1)$
\end{itemize}
\end{frame}

\begin{frame}{LDA hypothetical document generation}
As only one word occurs at each location in a document, 
    \vspace{0.2cm}
\begin{center}
{\mbox{\boldmath{$\sum_{i=1}^{K}Z_{ni} = \sum_{j=1}^{V}w_{nj} = 1$}}}.
\end{center}
%\end{enumerate}
\end{frame}



\section{K optimization methods}
\begin{frame}{K optimization methods}
\end{frame}


\begin{frame}{Perplexity (P)}
 
\begin{itemize}
\item Blei et al. (2003)
\item Algebraic inverse of geometric mean per-word  likelihood
\item Lower perplexity scores indicate better performance
\item Metric intrinsic to the LDA modeling process
\end{itemize}
%\end{frame}

%\begin{frame}{Perplexity (P)}
$$	
P = \exp \left\{ -\frac{\sum_{d=1}^{D_M} \log P({\bf w_d})}{\sum_{d=1}^{D_M} N_d} \right\}
$$
\end{frame}



\begin{frame}{Perplexity (P)}
\begin{figure}[htb!]
\includegraphics[width=8cm,height=5cm]{perplexity.png}
\end{figure}
\footnotetext[1]{Blei et al., 2003}
\end{frame}


\begin{frame}{Change in perplexity (dP)}
\begin{itemize}
\item Zhao et al. (2015)
\item Observed to be more stable and efficient as compared with raw peplexity
\end{itemize}
%\end{frame}

%\begin{frame}{Change in perplexity (dP)}

$$
dP = \frac{ P_{i+1} - P_{i}}{K_{i+1} - K_{i}}
$$
\end{frame}


\begin{frame}{Change in perplexity (dP)}
\begin{figure}[htb!]
\includegraphics[width=8cm,height=5cm]{dPerplexity.png}
\end{figure}
\footnotetext[1]{Zhao et al., 2015}
\end{frame}



\begin{frame}{2nd derivative of perplexity (d2P)}
\begin{itemize}
\item The zero line clarifies the change point described by Zhao et al.
\end{itemize}
\end{frame}

\begin{frame}{2nd derivative of perplexity (d2P)}
\begin{figure}[htb!]
\includegraphics[width=8cm,height=5cm]{d2Perplexity.png}
\end{figure}
\end{frame}



\begin{frame}{Goodness Of Fit (GOF)}
\begin{itemize}
\item Bowman, Chen and George (in press)
\item Statistical analysis of model output
\item Proportion of documents significantly different than random as metric to evaluate K

\item $H_{0}$: Topic assignment not different from random (uniform) assignment
\end{itemize}
\end{frame}

\begin{frame}{Goodness Of Fit (GOF)}
\begin{figure}[htb!]
\includegraphics[width=8cm,height=5cm]{GOF.png}
\end{figure}
\footnotetext[1]{Bowman, Chen and George (in press)}
\end{frame}




\begin{frame}{Pointwise Mutual Information (PMI)}
\begin{columns}
\column{.3\textwidth}
\begin{figure}[htb!]
\includegraphics[width=3cm,height=1.25cm]{PMI_formula.png}
\end{figure}
\column{.7\textwidth}
\begin{itemize}
\item Newman et al. (2010)
\item Extrinsic to modeling process
\item Corroborated by human judgment
\end{itemize}
\end{columns}
\footnotetext[1]{Image credit: http://lintool.github.io/UMD-courses.html}
\end{frame}


\begin{frame}{Pointwise Mutual Information (PMI)}
\begin{figure}[htb!]
\includegraphics[width=8cm,height=5cm]{PMI.png}
\end{figure}
\footnotetext[1]{Naraula et al., 2013 }
\end{frame}

\begin{frame}{Pointwise Mutual Information per K (PMI / K)}
\begin{figure}[htb!]
\includegraphics[width=8cm,height=5cm]{PMI_K.png}
\end{figure}
\end{frame}

\begin{frame}{Change in Pointwise Mutual Information per K (PMI / K)}
\begin{figure}[htb!]
\includegraphics[width=8cm,height=5cm]{dPMI_K.png}
\end{figure}
\end{frame}




%\section{Conclusions}
\begin{frame}{Conclusion: Optimization method comparisons}
\begin{itemize}
\item Methods of optimizing $K$ do not converge
\item Different optimization methods will be appropriate for differing purposes
\item Judgment is called for
\end{itemize}
\end{frame}

\begin{frame}{Thank you}
\begin{center}$\sim$ $Thank$  $you$  $\sim$
\end{center}
\end{frame}



\end{document}




